\documentclass {book}

\usepackage[T1]{fontenc}
\usepackage{textcomp}
\usepackage{amsmath}
\usepackage{color}
\usepackage{xcolor}

\newcommand{\bordercolor}[2]{%
	\colorlet{currentcolor}{.}%
	{\color{#1}%
		\fbox{\color{currentcolor}#2}}%
}


\definecolor{lightgrey}{HTML}{EEEEEE}
\definecolor{darkgrey}{HTML}{333333}
\definecolor{darkyellow}{HTML}{e5bb31}
\definecolor{stringcolor}{HTML}{aa1111}
\definecolor{numbercolor}{HTML}{116644}
\definecolor{defcolor}{HTML}{770088}
\definecolor{boolcolor}{HTML}{221199}

\usepackage[most]{tcolorbox}

\tcbset{
	colback=lightgrey,
	colframe=white,
	arc=0pt,outer arc=0pt,
}

%\pagenumbering{roman}

\begin {document}

	\title{JavaScript Basics For Absolute Beginners}
	\author{Mohammed Abdulhady}
	\date{\today}
	\maketitle
	\tableofcontents
	
	
	
	\chapter{Introduction}
	
	\section{History of JavaScript}
	JavaScript was created in 1995 by Brendan Eich who worked at Netscape. JavaScript was originally named Mocha by Marc Andreessen, the founder of Netscape. The name was changed to LiveScript and then to JavaScript. It was some kind of a marketing move because of the popularity of Java at that time.
	
	\section{What is JavaScript?}
	JavaScript is an object-oriented programming language that is used to make web pages interactive and creating web applications. JavaScript is an interpreted language, it needs an interpreter to translate the codes and executes the program. The interpreter, which is the browser in our case, executes instructions directly line by line without previously compiling it into machine language.\\\\
	Many people confused between Java (developed in the 1990s at Sun Microsystems) and JavaScript. They are completely different. Only the names are similar.
	
	\section{Why JavaScript}
	JavaScript is a very simple, powerful and popular programming language. It has become an essential technology along with HTML and CSS.\\
	You will need to learn JavaScript if you want to get into web development, if you are planning to be a front-end or you can even use it for back-end development.\\
	As o this moment, JavaScript has extended to desktop apps development, mobile apps development and games development.
	It has become a very important and useful skill to master.
	
	\chapter{Numbers}
	\section{What Are Numbers?}
	\textbf{Numbers} are the values that you can use mathematical operations on.\\
	There is not any special syntax for numbers in JavaScript. You can just write them straight into JavaScript.
	
	\begin{tcolorbox}
		\scriptsize
		\textcolor{darkgrey}{Example}\\
		
		\texttt{\textcolor{numbercolor}{123456};}

	\end{tcolorbox}
	
	\section{Decimals and fractions}
	Numbers in JavaScript can be written with, or without, decimals.
	Unlike any other programming language, JavaScript does not distinguish between numbers. It has only one type of numbers. So, you can use whole number and decimals together without worrying or having to convert from one to another.
	
	\begin{tcolorbox}
		\scriptsize
		\textcolor{darkgrey}{Example}\\
		
		\texttt{\textcolor{numbercolor}{1 + 2.345};}\\
		
		\textcolor{darkgrey}{Output}\\
		
		\texttt{\textcolor{numbercolor}{3.345}}
		
	\end{tcolorbox}
	
	\section{Negative Numbers}
	If you want to make any number negative, you can do that by placing the \bordercolor{darkyellow}{-} operator in front of it.
	
	\begin{tcolorbox}
		\scriptsize
		\textcolor{darkgrey}{Example}\\
		
		\texttt{\textcolor{numbercolor}{-100};}\\
		
		\textcolor{darkgrey}{Output}\\
		
		\texttt{\textcolor{numbercolor}{-100}}
		
	\end{tcolorbox}
	
	Also, you can get a negative number by subtracting a big number from a smaller one.
	
	\begin{tcolorbox}
		\scriptsize
		\textcolor{darkgrey}{Example}\\
		
		\texttt{\textcolor{numbercolor}{3 - 100};}\\
		
		\textcolor{darkgrey}{Output}\\
		
		\texttt{\textcolor{numbercolor}{-97}}
	\end{tcolorbox}
	
	\section{Arithmetic Operators}
	\textbf{Operators} are those symbols that you use between two or more values to perform different operations such as addition, subtraction and more.\\
	In this book, we will focus on 	the ones that we think you will see most often.
	
	\subsection{Addition}
	the \bordercolor{darkyellow}{+} operator is used to add two or more numbers.
	\begin{tcolorbox}
		\scriptsize
		\textcolor{darkgrey}{Example}\\
		
		\texttt{\textcolor{numbercolor}{3 + 2 + 1};}\\
		
		\textcolor{darkgrey}{Output}\\
		
		\texttt{\textcolor{numbercolor}{6}}
	\end{tcolorbox}
	
	\subsection{Subtraction}
	the \bordercolor{darkyellow}{-} operator is used to subtract numbers from each others.
	\begin{tcolorbox}
		\scriptsize
		\textcolor{darkgrey}{Example}\\
		
		\texttt{\textcolor{numbercolor}{3 - 2 - 1};}\\
		
		\textcolor{darkgrey}{Output}\\
		
		\texttt{\textcolor{numbercolor}{0}}
	\end{tcolorbox}
	
	
	\subsection{Multiplication}
	the \bordercolor{darkyellow}{*} operator is used to multiply two or more numbers. notice that we use \bordercolor{darkyellow}{*} instead of \bordercolor{darkyellow}{$\times$} symbol that is commonly used in math.
	\begin{tcolorbox}
		\scriptsize
		\textcolor{darkgrey}{Example}\\
		
		\texttt{\textcolor{numbercolor}{3 * 2};}\\
		
		\textcolor{darkgrey}{Output}\\
		
		\texttt{\textcolor{numbercolor}{6}}
	\end{tcolorbox}
	
	\subsection{Division}
	the \bordercolor{darkyellow}{/} operator is used to divide numbers by numbers. notice that we use \bordercolor{darkyellow}{/} instead of \bordercolor{darkyellow}{$\div$} symbol that is commonly used in math.
	\begin{tcolorbox}
		\scriptsize
		\textcolor{darkgrey}{Example}\\
		
		\texttt{\textcolor{numbercolor}{6 / 2};}\\
		
		\textcolor{darkgrey}{Output}\\
		
		\texttt{\textcolor{numbercolor}{3}}
	\end{tcolorbox}
	
	\subsection{Modulus}
	the \bordercolor{darkyellow}{\%} operator is used to get the division remainder of two numbers.
	\begin{tcolorbox}
		\scriptsize
		\textcolor{darkgrey}{Example}\\
		
		\texttt{\textcolor{numbercolor}{5 \% 2};}\\
		
		\textcolor{darkgrey}{Output}\\
		
		\texttt{\textcolor{numbercolor}{1}}
	\end{tcolorbox}
	
	
	\subsection{Increment}
	the \bordercolor{darkyellow}{++} operator is used to increment the number by 1.
	\begin{tcolorbox}
		\scriptsize
		\textcolor{darkgrey}{Example}\\
		
		\texttt{\textcolor{numbercolor}{5++};}\\
		\texttt{\textcolor{numbercolor}{5+1};}\\
		
		\textcolor{darkgrey}{Output}\\
		
		\texttt{\textcolor{numbercolor}{6}}\\
		\texttt{\textcolor{numbercolor}{6}}
	\end{tcolorbox}
	
	
	\subsection{Decrement}
	the \bordercolor{darkyellow}{- -} operator is used to decrement the number by 1.
	\begin{tcolorbox}
		\scriptsize
		\textcolor{darkgrey}{Example}\\
		
		\texttt{\textcolor{numbercolor}{5---};}\\
		\texttt{\textcolor{numbercolor}{5-1};}\\
		
		\textcolor{darkgrey}{Output}\\
		
		\texttt{\textcolor{numbercolor}{4}}\\
		\texttt{\textcolor{numbercolor}{4}}
	\end{tcolorbox}
	
	\section{Ordering and Grouping}
	JavaScript expressions follow the \textbf{\textcolor{darkyellow}{order of operations}} , so even if the \bordercolor{darkyellow}{+} and \bordercolor{darkyellow}{-} operators come first in the following example, the \bordercolor{darkyellow}{*} and \bordercolor{darkyellow}{/} operators will be performed first between the second and third number.
	\begin{tcolorbox}
		\scriptsize
		\textcolor{darkgrey}{Example}\\
		
		\texttt{\textcolor{numbercolor}{5 + 10 * 2};}\\
		\texttt{\textcolor{numbercolor}{5 - 10 / 2};}\\
		
		\textcolor{darkgrey}{Output}\\
		
		\texttt{\textcolor{numbercolor}{25}}\\
		\texttt{\textcolor{numbercolor}{0}}
	\end{tcolorbox}
	
	\noindent \\However, you can have more control over the order of operations using the grouping operator \bordercolor{darkyellow}{()}.\\
	Using these operators to group other values and operations will force JavaScript to perform the grouped operations first, no matter what the ordering is.
	
	\begin{tcolorbox}
		\scriptsize
		\textcolor{darkgrey}{Example}\\
		
		\texttt{\textcolor{numbercolor}{(5 + 10) * 2};}\\
		\texttt{\textcolor{numbercolor}{(5 - 10) / 2};}\\
		
		\textcolor{darkgrey}{Output}\\
		
		\texttt{\textcolor{numbercolor}{30}}\\
		\texttt{\textcolor{numbercolor}{-2.5}}
	\end{tcolorbox}
	
	
	
	\chapter{Strings}
	\section{What Are Strings ?}

	\textbf{Strings} are values made up with series of characters. They can be made of letters, symbols, numbers or anything.\\\\
	A JavaScript String must be contained inside a pair of double quotation \bordercolor{darkyellow}{" "} or ever single quotation \bordercolor{darkyellow}{\textquotesingle\ \textquotesingle}.

	\begin{tcolorbox}
		\scriptsize
		\textcolor{darkgrey}{Example}\\
		
		\texttt{\textcolor{stringcolor}{\textquotesingle This is our string.\textquotesingle};}
		
		\texttt{\textcolor{stringcolor}{"This is our string."};}
		
	\end{tcolorbox}

	\section{Enclosing quotation marks}
	Let us say you want to use the quotation marks inside your string. You can not do this normally like any other character, because quotation marks are one of the \textbf{\textcolor{darkyellow}{special characters}} family.\\
	To solve that problem, you have to easy options.\\
	First option, you will need to use opposite quotation marks inside and outside. This means for each string that include single quotes needs to use double quotes and vice versa.
	
	\begin{tcolorbox}
		\scriptsize
		\textcolor{darkgrey}{Example}\\
		
		\texttt{\textcolor{stringcolor}{"I\textquotesingle ve eaten an apple."};}
		
		\texttt{\textcolor{stringcolor}{\textquotesingle I said "I have eaten an apple"\textquotesingle};}
		
	\end{tcolorbox}
	
	\noindent \\\\\\Second option, you can use a backslash \bordercolor{darkyellow}{\small{\textbackslash}} right before each quote inside the string. This makes the JavaScript know that you want to use a \textbf{\textcolor{darkyellow}{special characters}}.
	
	\begin{tcolorbox}
		\scriptsize
		\textcolor{darkgrey}{Example}\\
		
		\texttt{\textcolor{stringcolor}{\textquotesingle I\textbackslash\textquotesingle ve eaten an apple.\textquotesingle};}
		
		\texttt{\textcolor{stringcolor}{"I said \textbackslash"I have eaten an apple\textbackslash""};}
		
	\end{tcolorbox}
	
	
	\section{Methods and properties}
	Strings in JavaScript have their own predefined list of operations you can perform to string. They are call "methods and properties".\\
	Here are some of the most helpful and commonly used :-
	
	\subsection{Length}
	The string \bordercolor{darkyellow}{length} is a property that returns the number of characters that the string has.
	\begin{tcolorbox}
		\scriptsize
		\textcolor{darkgrey}{Example}\\
		
		\texttt{\textcolor{stringcolor}{"What is my length?"}.length;}\\
		
		\textcolor{darkgrey}{Output}\\
		
		\texttt{\textcolor{numbercolor}{18}}
	\end{tcolorbox}
	
	\subsection{toLowerCase}
	The String \bordercolor{darkyellow}{toLowerCase()} Method returns a copy of the string but with all capital letters converted to small letters.
	\begin{tcolorbox}
		\scriptsize
		\textcolor{darkgrey}{Example}\\
		
		\texttt{\textcolor{stringcolor}{"I aM A StRiNg !!"}.toLowerCase();}\\
		
		\textcolor{darkgrey}{Output}\\
		
		\texttt{\textcolor{stringcolor}{"i am a string !!"}}
	\end{tcolorbox}
	
	
	\subsection{toUpperCase}
	The String \bordercolor{darkyellow}{toUpperCase()} Method returns a copy of the string but with all small letters converted to capital letters.
	\begin{tcolorbox}
		\scriptsize
		\textcolor{darkgrey}{Example}\\
		
		\texttt{\textcolor{stringcolor}{"I aM A StRiNg !!"}.toUpperCase();}\\
		
		\textcolor{darkgrey}{Output}\\
		
		\texttt{\textcolor{stringcolor}{"I AM A STRING !!"}}
	\end{tcolorbox}
	
	\subsection{Trim}
	The \bordercolor{darkyellow}{trim()} method cuts off any whitespaces at the begin and end of the string.
	\begin{tcolorbox}
		\scriptsize
		\textcolor{darkgrey}{Example}\\
		
		\texttt{\textcolor{stringcolor}{"\ \ \ \ \ Trim me, please\ \ "}.trim();}\\
		
		\textcolor{darkgrey}{Output}\\
		
		\texttt{\textcolor{stringcolor}{"Trim me, please"}}
	\end{tcolorbox}
	
	\subsection{Replace}
	The String \bordercolor{darkyellow}{replace()} method replaces a specified value with another value in a string and returns the result.
	\begin{tcolorbox}
		\scriptsize
		\textcolor{darkgrey}{Example}\\
		
		\texttt{\textcolor{stringcolor}{"I love red oranges"}.replace("oranges", "apples");}\\
		
		\textcolor{darkgrey}{Output}\\
		
		\texttt{\textcolor{stringcolor}{"I love red apples"}}
	\end{tcolorbox}
	
	\chapter{Booleans}
	\textbf{Booleans} are the values that can only be one of two things: either \bordercolor{darkyellow}{true} or \bordercolor{darkyellow}{false}.\\
	It's really useful to store booleans in \textbf{\textcolor{darkyellow}{variables}} to keep track of their value and change them as you like.
	
	\begin{tcolorbox}
		\scriptsize
		\textcolor{darkgrey}{Example}\\
		
		\texttt{\textcolor{defcolor}{var} \textcolor{blue}{flag} = \textcolor{boolcolor}{true};}\\
		
		\texttt{\textcolor{blue}{flag} = \textcolor{boolcolor}{false};}\\
		
		\texttt{\textcolor{blue}{flag};}\\
		
		\textcolor{darkgrey}{Output}\\
		
		\texttt{\textcolor{boolcolor}{false}}
	\end{tcolorbox}
	
	\noindent Booleans are also essential for \textbf{\textcolor{darkyellow}{conditional}} work.
	We will discuss that later.
	
	\chapter{Variables}
	\textbf{Variables} are named values that we use to store any type of JavaScript Value.\\
	The following example is how to declare a variable:-W
	
	\begin{tcolorbox}
		\scriptsize
		\textcolor{darkgrey}{Example}\\
		
		\texttt{\textcolor{defcolor}{var} \textcolor{blue}{myNumber} = \textcolor{numbercolor}{100};}
		
	\end{tcolorbox}
	
	\noindent And this is what is happening in the example above:-
	
	\begin{itemize}
		\item \bordercolor{darkyellow}{var} is a keyword you use to declare a variable.
		\item \bordercolor{darkyellow}{myNumber} is the name of the variable.
		\item \bordercolor{darkyellow}{=} is an \textbf{\textcolor{darkyellow}{operator}} used for assigning the value to the variable.
		\item \bordercolor{darkyellow}{100} is the value you want to store at the variable.
		
	\end{itemize}
	\section{Using Variables}
	After declaring your variable, you can reference it by its name anywhere after the declaration in your code for further use.
	
	\begin{tcolorbox}
		\scriptsize
		\textcolor{darkgrey}{Example}\\
		
		\texttt{\textcolor{defcolor}{var} \textcolor{blue}{myNumber} = \textcolor{numbercolor}{100};}\\
		
		\texttt{\textcolor{blue}{myNumber} + \textcolor{numbercolor}{50};}\\
		
		\textcolor{darkgrey}{Output}\\
		
		\texttt{\textcolor{numbercolor}{150}}
	\end{tcolorbox}
	
	\noindent You can even use variable while declaring another one.
	
	\begin{tcolorbox}
		\scriptsize
		\textcolor{darkgrey}{Example}\\
		
		\texttt{\textcolor{defcolor}{var} \textcolor{blue}{x} = \textcolor{numbercolor}{100};}\\
		
		\texttt{\textcolor{defcolor}{var} \textcolor{blue}{y} = \textcolor{blue}{x} + \textcolor{numbercolor}{50};}\\
		
		\texttt{\textcolor{blue}{y};}\\
		
		\textcolor{darkgrey}{Output}\\
		
		\texttt{\textcolor{numbercolor}{150}}
		
	\end{tcolorbox}
	
	\section{Reassigning Variables}
	What if you needed to change a value of a variable you have declared before ?! is it possible ?!\\
	Well, of course it is !!
	
	\begin{tcolorbox}
		\scriptsize
		\textcolor{darkgrey}{Example}\\
		
		\texttt{\textcolor{defcolor}{var} \textcolor{blue}{color} = \textcolor{stringcolor}{"red"};}\\
		
		\texttt{\textcolor{blue}{color} = \textcolor{stringcolor}{"blue"};}\\
		
		\texttt{\textcolor{blue}{color};}\\
		
		\textcolor{darkgrey}{Output}\\
		
		\texttt{\textcolor{stringcolor}{"blue"}}
		
	\end{tcolorbox}
	
	\section{Naming Variables}
	Naming variable is really easy and flexible as long as you follow these simple rules:-
	\begin{itemize}
		\item You can only start them with a letter, underscore, or a dollar sign.
		\item After the first letter, you can use numbers as you want.
		\item You can not use any JavaScript \textbf{\textcolor{darkyellow}{reserved}} keyword.
		
	\end{itemize}
	 
	Keeping that in your mind, these are some valid variable names.
	\begin{tcolorbox}
		\scriptsize
		\textcolor{darkgrey}{Example}\\
		
		\texttt{\textcolor{defcolor}{var} \textcolor{blue}{camelCase} = \textcolor{stringcolor}{"first word lowercase, then everything goes uppercase."};}\\
		
		\texttt{\textcolor{defcolor}{var} \textcolor{blue}{book2write} = \textcolor{stringcolor}{"My JavaScript mini-book for absolute beginners."};}\\
		
		\texttt{\textcolor{defcolor}{var} \textcolor{blue}{I\_FEEL\_EXCITED} = \textcolor{boolcolor}{false};}\\
		
		\texttt{\textcolor{defcolor}{var} \textcolor{blue}{\$\_\$} = \textcolor{stringcolor}{"Money eyes"};}\\
		
		\texttt{\textcolor{defcolor}{var} \textcolor{blue}{\_1000000\$\_} = \textcolor{stringcolor}{"I wish that was my salary"};}\\
		
	\end{tcolorbox}
	\noindent And these are some invalid variable name. Can you try to tell what is wrong with each ?
	
	\begin{tcolorbox}
		\scriptsize
		\textcolor{darkgrey}{Example}\\
		
		\texttt{\textcolor{defcolor}{var} \textcolor{blue}{total\%} = \textcolor{numbercolor}{100};}\\
		
		\texttt{\textcolor{defcolor}{var} \textcolor{blue}{2books2write} = \textcolor{boolcolor}{false};}\\
		
		\texttt{\textcolor{defcolor}{var} \textcolor{blue}{function} = \textcolor{stringcolor}{"myFunction"};}\\
		
		\texttt{\textcolor{defcolor}{var} \textcolor{blue}{money+money} = \textcolor{stringcolor}{"a lot of money"};}\\
		
	\end{tcolorbox}
	
	\noindent In JavaScript, Variable names are case-sensitive, so \bordercolor{darkyellow}{myVar},
	\bordercolor{darkyellow}{MyVar},
	\bordercolor{darkyellow}{MYVAR},
	\bordercolor{darkyellow}{myvar} are all different variables. However, it is a good practice to avoid naming variables so similarly. Things could really get mixed up.
	
	
	\chapter{Functions}
	\textbf{Variables} are named blocks of code that you can call by its name and reuse anywhere where you like.\\
	This is how to declare a function:-
	\begin{tcolorbox}
		\scriptsize
		\textcolor{darkgrey}{Example}\\
		
		\texttt{\textcolor{defcolor}{function} \textcolor{blue}{addNumbers} (\textcolor{blue}{x}, \textcolor{blue}{y}) \{\\
			\parbox{10px}{\ }\textcolor{defcolor}{return} \textcolor{blue}{x} + \textcolor{blue}{y};\\
		\}}
	
	\end{tcolorbox}
	\noindent A lot is going on in the example above. So, let us have a look at each part separately.
	
	\begin{itemize}
		
		\item \bordercolor{darkyellow}{function} is a keyword for declaring a function.
		\item \bordercolor{darkyellow}{addNumbers} is the name of the function, which is customizable just like \textbf{\textcolor{darkyellow}{reserved}} names.
		\item \bordercolor{darkyellow}{(x, y)} are parameters that you pass to the function to perform different operation on. \textbf{\textcolor{darkyellow}{return}} is a keyword that exits the function and share only 1 value outside.
		
	\end{itemize}
	
	In our case the \textbf{\textcolor{darkyellow}{return}} will share sum of \textbf{\textcolor{darkyellow}{x}} and \textbf{\textcolor{darkyellow}{y}}.
	
	\section{Using Functions}
	Once you define your function, you can call it by referencing its name with parentheses () right after it.
	Notice that the following function does nt have any parameters.
	\begin{tcolorbox}
		\scriptsize
		\textcolor{darkgrey}{Example}\\
		
		\texttt{\textcolor{defcolor}{function} \textcolor{blue}{sayHello}() \{\\
			\parbox{10px}{\ }\textcolor{defcolor}{return} \textcolor{stringcolor}{"Hello !!"};\\
		\}}\\
		
		\texttt{\textcolor{blue}{sayHello}();}\\
		
		\textcolor{darkgrey}{Output}\\\\
		\texttt{\textcolor{stringcolor}{"Hello !!"}}
		
	\end{tcolorbox}
	
	\noindent If the function has any parameters, you will need to pass the values that will be represented by the parameters names inside the function.
	\begin{tcolorbox}
		\scriptsize
		\textcolor{darkgrey}{Example}\\
		
		\texttt{\textcolor{defcolor}{function} \textcolor{blue}{sayHello}(\textcolor{blue}{name}) \{\\
			\parbox{10px}{\ }\textcolor{defcolor}{return} \textcolor{stringcolor}{"Hello, "} + \textcolor{blue}{name};\\
		\}}
	
		\texttt{\textcolor{blue}{sayHello}(\textcolor{stringcolor}{"Mohamed"});}\\
		
		\textcolor{darkgrey}{Output}\\\\
		\texttt{\textcolor{stringcolor}{"Hello, Mohamed"}}
		
	\end{tcolorbox}
	
	\chapter{Conditionals}
	\textbf{Conditionals} can control the behavior of your code. It determine whether or not a block of code runs according to a specified condition.
	
	\section{If}
	\textbf{Conditionals} statement is the most common type of conditionals. It the code inside it only runs if the condition inside its parentheses is \textbf{true}.
	
	\noindent This is how an \bordercolor{darkyellow}{If} statement should look like:-
	
	\begin{tcolorbox}
		\scriptsize
		\textcolor{darkgrey}{Example}\\
		
		\texttt{\textcolor{defcolor}{if} (\textcolor{numbercolor}{1} < \textcolor{numbercolor}{10}) \{\\
			\parbox{10px}{\ }\textcolor{defcolor}{var} \textcolor{blue}{output} = \textcolor{stringcolor}{"Code Block"};\\
		\}}
		
		\texttt{\textcolor{blue}{output};}\\
		
		\textcolor{darkgrey}{Output}\\
		
		\texttt{\textcolor{stringcolor}{"Code Block"}}
	
	\end{tcolorbox}
	\noindent Pretty sure you understand it all. But, let us discuss every part just in case you missed something.
	
	\begin{itemize}
		
		\item \bordercolor{darkyellow}{if} is the keyword to start the conditional statement.
		\item \bordercolor{darkyellow}{(1 < 10)} is the condition, which is \bordercolor{darkyellow}{true} in this case.	
		\item The curly braces \bordercolor{darkyellow}{\{\}} are the containers of the code block.
		\item Because our condition is true, the variable \bordercolor{darkyellow}{output} will be assigned to the value \bordercolor{darkyellow}{"Code Block"} 
		
	\end{itemize}
	
	\section{else}
	However, you can extend your \bordercolor{darkyellow}{if} statement with an \bordercolor{darkyellow}{else} statement, which runs only if the condition was \textbf{false}:-
	\begin{tcolorbox}
		\scriptsize
		\textcolor{darkgrey}{Example}\\
		
		\texttt{\textcolor{defcolor}{if} (\textcolor{boolcolor}{false}) \{\\
			\parbox{10px}{\ }\textcolor{defcolor}{var} \textcolor{blue}{output} = \textcolor{stringcolor}{"If Block"};\\
		\} \textcolor{defcolor}{else} \{\\
			\parbox{10px}{\ }\textcolor{defcolor}{var} \textcolor{blue}{output} = \textcolor{stringcolor}{"Else Block"};\\
		\}}
		
		\texttt{\textcolor{blue}{output};}\\
		
		\textcolor{darkgrey}{Output}\\
		
		\texttt{\textcolor{stringcolor}{"Else Block"}}
		
	\end{tcolorbox}
	\noindent You can also use multiple \bordercolor{darkyellow}{else if} statements, if you have more conditions to use.
	\begin{tcolorbox}
		\scriptsize
		\textcolor{darkgrey}{Example}\\
		
		\texttt{\textcolor{defcolor}{if} (\textcolor{numbercolor}{1} > \textcolor{boolcolor}{5}) \{\\
			\parbox{10px}{\ }\textcolor{defcolor}{var} \textcolor{blue}{output} = \textcolor{stringcolor}{"If Block"};\\	
		\} \textcolor{defcolor}{else if} (\textcolor{numbercolor}{10} > \textcolor{boolcolor}{5}) \{\\
			\parbox{10px}{\ }\textcolor{defcolor}{var} \textcolor{blue}{output} = \textcolor{stringcolor}{"First Else If Block"};\\
		\} \textcolor{defcolor}{else if} (\textcolor{numbercolor}{100} > \textcolor{boolcolor}{5}) \{\\
		\parbox{10px}{\ }\textcolor{defcolor}{var} \textcolor{blue}{output} = \textcolor{stringcolor}{"Second Else If Block"};\\
		\} \textcolor{defcolor}{else} \{\\
			\parbox{10px}{\ }\textcolor{defcolor}{var} \textcolor{blue}{output} = \textcolor{stringcolor}{"Else Block"};\\
		\}}
		
		\texttt{\textcolor{blue}{output};}\\
		
		\textcolor{darkgrey}{Output}\\
		
		\texttt{\textcolor{stringcolor}{"First Else If Block"}}
		
	\end{tcolorbox}
	\noindent Note that only the first \bordercolor{darkyellow}{else if} statement that passes the condition runs, regardless of the following statements.
	
	\chapter{Arrays}
	\textbf{Arrays} are containers that can hold multiple values inside. These values call \textbf{elements}.
	\begin{tcolorbox}
		\scriptsize
		\textcolor{darkgrey}{Example}\\
		
		\texttt{\textcolor{defcolor}{var} \textcolor{blue}{food} = [\textcolor{stringcolor}{"meat"}, \textcolor{stringcolor}{"rice"}];\\
		\textcolor{blue}{food};}\\
		
		\textcolor{darkgrey}{Output}\\
		
		\texttt{[\textcolor{stringcolor}{"meat"}, \textcolor{stringcolor}{"rice"}]}
		
	\end{tcolorbox}
	
	\noindent In JavaScript, array elements do not have to be the same value type. They can even be another array.
	
	\begin{tcolorbox}
		\scriptsize
		\textcolor{darkgrey}{Example}\\
		
		\texttt{\textcolor{defcolor}{var} \textcolor{blue}{items} = [\textcolor{stringcolor}{"book"}, \textcolor{numbercolor}{100}, \textcolor{boolcolor}{false}, [\textcolor{numbercolor}{5}, \textcolor{stringcolor}{"pen"}], \textcolor{numbercolor}{1.5}];\\
			\textcolor{blue}{items};}\\
		
		\textcolor{darkgrey}{Output}\\
		
		\texttt{[\textcolor{stringcolor}{"book"}, \textcolor{numbercolor}{100}, \textcolor{boolcolor}{false}, [\textcolor{numbercolor}{5}, \textcolor{stringcolor}{"pen"}], \textcolor{numbercolor}{1.5}]}
		
	\end{tcolorbox}
	
	\section{Accessing Elements}
	To access elements inside an array, you need to use the array name followed by square brackets and the number of the element inside the brackets.\\
	Keep in mind that JavaScript numbering start with 0 not 1. So, the first element is number 0.
	
	
	\begin{tcolorbox}
		\scriptsize
		\textcolor{darkgrey}{Example}\\
		
		\texttt{\textcolor{defcolor}{var} \textcolor{blue}{colors} = [\textcolor{stringcolor}{"black"}, \textcolor{stringcolor}{"white"}];\\
			\textcolor{blue}{colors}[\textcolor{numbercolor}{0}];}\\
		
		\textcolor{darkgrey}{Output}\\
		
		\texttt{\textcolor{stringcolor}{"black"}}
		
	\end{tcolorbox}
	
	This can also work for setting values to to elements, not just getting them.
	
	\begin{tcolorbox}
		\scriptsize
		\textcolor{darkgrey}{Example}\\
		
		\texttt{\textcolor{defcolor}{var} \textcolor{blue}{colors} = [\textcolor{stringcolor}{"black"}, \textcolor{stringcolor}{"white"}, \textcolor{stringcolor}{"gren"}];\\
			\textcolor{blue}{colors}[\textcolor{numbercolor}{2}] = \textcolor{stringcolor}{"green"};\\
			\textcolor{blue}{colors};}\\
		
		\textcolor{darkgrey}{Output}\\
		
		\texttt{[\textcolor{stringcolor}{"black"}, \textcolor{stringcolor}{"white"}, \textcolor{stringcolor}{"green"}]}
		
	\end{tcolorbox}
	
	\section{Methods And Properties}
	Arrays also have their own predefined list of operations you can perform to string. They are call "methods and properties".\\
	Here are some of the most helpful and commonly used :-
	
	\subsection{length}
	The \bordercolor{darkyellow}{length} property is used to get the number of elements inside an array.
	
	\begin{tcolorbox}
		\scriptsize
		\textcolor{darkgrey}{Example}\\
		
		\texttt{[\textcolor{numbercolor}{1}, \textcolor{numbercolor}{2}, \textcolor{numbercolor}{3}].length;}\\
		
		\textcolor{darkgrey}{Output}\\
		
		\texttt{\textcolor{numbercolor}{3}}
		
	\end{tcolorbox}
	
	\subsection{concat()}
	\bordercolor{darkyellow}{concat()} is a method that returns a new array that combines the values of the other two array.
	
	\begin{tcolorbox}
		\scriptsize
		\textcolor{darkgrey}{Example}\\
		
		\texttt{[\textcolor{stringcolor}{"red"}].concat([\textcolor{stringcolor}{"green"}, \textcolor{stringcolor}{"blue"}]);}\\
		
		\textcolor{darkgrey}{Output}\\
		
		\texttt{[\textcolor{stringcolor}{"red"}, \textcolor{stringcolor}{"green"}, \textcolor{stringcolor}{"blue"}]}
		
	\end{tcolorbox}
	
	\subsection{push()}
	\bordercolor{darkyellow}{push()} is a method that appends an element to the end of an array and returns the new length.
	
	\begin{tcolorbox}
		\scriptsize
		\textcolor{darkgrey}{Example}\\
		
		\texttt{[\textcolor{stringcolor}{"red"}, \textcolor{stringcolor}{"green"}, \textcolor{stringcolor}{"blue"}].push(\textcolor{stringcolor}{"yellow"});}\\
		
		\textcolor{darkgrey}{Output}\\
		
		\texttt{\textcolor{numbercolor}{4}}
		
	\end{tcolorbox}
	
	\subsection{pop()}
	\bordercolor{darkyellow}{pop()} is a method that removes the last element from the array and returns that element.
	
	\begin{tcolorbox}
		\scriptsize
		\textcolor{darkgrey}{Example}\\
		
		\texttt{[\textcolor{stringcolor}{"red"}, \textcolor{stringcolor}{"green"}, \textcolor{stringcolor}{"blue"}, \textcolor{stringcolor}{"yellow"}].pop();}\\
		
		\textcolor{darkgrey}{Output}\\
		
		\texttt{\textcolor{stringcolor}{"yellow"}}
		
	\end{tcolorbox}
	
	
	\subsection{reverse()}
	\bordercolor{darkyellow}{reverse()} is a method that returns the array in an opposite order.
	
	\begin{tcolorbox}
		\scriptsize
		\textcolor{darkgrey}{Example}\\
		
		\texttt{[\textcolor{stringcolor}{"red"}, \textcolor{stringcolor}{"green"}, \textcolor{stringcolor}{"blue"}, \textcolor{stringcolor}{"yellow"}].reverse();}\\
		
		\textcolor{darkgrey}{Output}\\
		
		\texttt{[\textcolor{stringcolor}{"yellow"}, \textcolor{stringcolor}{"blue"}, \textcolor{stringcolor}{"green"}, \textcolor{stringcolor}{"red"}]}
		
	\end{tcolorbox}
	
	\chapter{Objects}
	\textbf{Objects} are containers that can contain multiple values of multiple types using \textbf{keys} to name these values.\\
	
	And this is how Object look like in JavaScript:-
	
	\begin{tcolorbox}
		\scriptsize
		\textcolor{darkgrey}{Example}\\
		
		\texttt{\textcolor{defcolor}{var} \textcolor{blue}{student} = \{\\
			\parbox{10px}{\ }name: \textcolor{stringcolor}{"John Cena"},\\
			\parbox{10px}{\ }age: \textcolor{numbercolor}{21},\\
			\parbox{10px}{\ }grade: \textcolor{numbercolor}{3}\\
		\};}
		
	\end{tcolorbox}
	
	\section{Getting Values}
	To get a value using keys, you have two options:-\\
	You can use a dot \bordercolor{darkyellow}{.} notation.
	
	\begin{tcolorbox}
		\scriptsize
		\textcolor{darkgrey}{Example}\\
		
		\texttt{\textcolor{defcolor}{var} \textcolor{blue}{student} = \{\\
			\parbox{10px}{\ }name: \textcolor{stringcolor}{"John Cena"},\\
			\parbox{10px}{\ }age: \textcolor{numbercolor}{21},\\
			\parbox{10px}{\ }grade: \textcolor{numbercolor}{3}\\
		\};\\
		\textcolor{blue}{student}.name;}\\
	
		\textcolor{darkgrey}{Output}\\
		\texttt{\textcolor{stringcolor}{"John Cena"}}
		
	\end{tcolorbox}
	
	Or, you can use the square brackets and the double quotations \bordercolor{darkyellow}{[""]} with the key inside them.
	
	\begin{tcolorbox}
		\scriptsize
		\textcolor{darkgrey}{Example}\\
		
		\texttt{\textcolor{defcolor}{var} \textcolor{blue}{student} = \{\\
			\parbox{10px}{\ }name: \textcolor{stringcolor}{"John Cena"},\\
			\parbox{10px}{\ }age: \textcolor{numbercolor}{21},\\
			\parbox{10px}{\ }grade: \textcolor{numbercolor}{3}\\
			\};\\
			\textcolor{blue}{student}[\textcolor{stringcolor}{"name"}];}\\
		
		\textcolor{darkgrey}{Output}\\
		\texttt{\textcolor{stringcolor}{"John Cena"}}
		
	\end{tcolorbox}
	
	\section{Setting Values}
	
	To set values or overwrite them using keys. you can use the two options we discussed before.
	
	\begin{tcolorbox}
		\scriptsize
		\textcolor{darkgrey}{Example}\\
		
		\texttt{\textcolor{defcolor}{var} \textcolor{blue}{student} = \{\\
			\parbox{10px}{\ }name: \textcolor{stringcolor}{"John Cena"},\\
			\parbox{10px}{\ }age: \textcolor{numbercolor}{21},\\
			\parbox{10px}{\ }grade: \textcolor{numbercolor}{3}\\
			\};\\
			\textcolor{blue}{student}[\textcolor{stringcolor}{"name"}] = \textcolor{stringcolor}{"Mohammed"};\\
			\textcolor{blue}{student}.age = \textcolor{numbercolor}{23};\\}

		
		\textcolor{darkgrey}{Output}\\
		\texttt{\{\\
			\parbox{10px}{\ }\textcolor{stringcolor}{"name"}: \textcolor{stringcolor}{"Mohammed"},\\
			\parbox{10px}{\ }\textcolor{stringcolor}{"age"}: \textcolor{numbercolor}{23},\\
			\parbox{10px}{\ }\textcolor{stringcolor}{"grade"}: \textcolor{numbercolor}{3}\\
		\};\\}
		
	\end{tcolorbox}
	
	\chapter{Outro}
	As for now, you can do some very simple app but you still can not really build these awesome applications you had in mind, YET. However, you should be familiar with JavaScript whenever you see it.\\
	
	\noindent We have not really got to the real JavaScript. That was just a small talk about the basics of JavaScript and pretty much every other high level programming languages. There is much more going on.\\
	
	\noindent If you like JavaScript by far (and I am sure you do now) you can start digging through the Internet to get deeply into the real JavaScript. There are many website that will really help you start. I personally recommend you \textbf{The W3School} or \textbf{The MDN Website}. They are covering almost every this about JavaScript and many other web-related programming languages.
	
	
	
	\begin{thebibliography}{9}
		\bibitem{javascriptwebsite} 
		JavaScript Website. \\\texttt{https://www.javascript.com/learn/javascript}
		
		\bibitem{w3schoolwebsite} 
		W3School. \\\texttt{https://www.w3schools.com/js}
		
		\bibitem{mdnwebsite} 
		Mozilla Developer Network.
		\\\texttt{https://developer.mozilla.org/en-US/docs/Web/JavaScript}
	\end{thebibliography}

		
\end {document}


